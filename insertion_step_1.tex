
% Optional arguments: Aligns the illustrations with respect to each other
\begin{tikzpicture}[remember picture]
    

    \begin{scope}
        % Create the hashtable
        \matrix[hashtable] (table) {
        % Row with keys
        67 & & & & & & & & & & \\
        % Row with indices
         & & & & & & & & & & \\
        };

        % Insert some descriptive text and position it correctly with respect to the table...
        \node[left = 1mm and 0 of table-1-1.west] (label1) {\small{Inserted keys $k$:}};
        \node[below = 0.5mm and 0 of label1.south] {\small{\hspace{1.35cm} Index $j$:}};
        \node[above right = 1mm and 0 of table-1-1.north west] {Hash table};
    
        % Define the key to be inserted
        \matrix[index, below left = 0mm and 0 of table, anchor = north east] (insertKey) {
            \small{Data} & \small{Key} \\
            $ \circ $ & $k =$ 67 \\
        };

        % Illustrate the operation
        % Use \phantom to add some "invisible" characters for spacing
        \matrix[operation, anchor = north west] () {
            \phantom{\small{Data}} & \phantom{\small{Key}} \\
            \phantom{\small{Data}} & \phantom{\small{Key}} \\
                        \small{\color{green}{Inserted\phantom{xx}:}}& \small{$h(67, 0) = 0$} \\
        };

        % Color boxes at insertion or collision 
        \begin{scope}[on background layer]
                        \fill[green] (table-1-1.north west) rectangle (table-1-1.south east);
        \end{scope}

        % Color the path from the (data, key) pair to calculated index in the hashtable
                        \draw[->, green] (insertKey-2-2) to[out = 0, in = -90] (table-2-1.south);
                        \draw[dashed] (insertKey-2-1) -- (insertKey-2-2.west) node[xshift = - 1.30cm, color = gray]{};

    \end{scope}

\end{tikzpicture}
